
\section{Review of Contraction Theory}

For your smooth start, we recommend you to begin with \cite{LOHMILLER:1998aa}.
The overview of contraction theory is presented in a review paper \cite{Tsukamoto:2021aa}.

\subsection{Basic Results of Contraction Theory for Deterministic Systems}

First, we start with the following deterministic systems:
\begin{equation}
    \ddtt{\bdx}
    = 
    \bdf(\bdx,t)
    ,
    \label{eq:sys}
\end{equation}
where $\bdf(\bdx,t)$ is an $n\times1$ sufficiently smooth non-linear vector function and $\bdx\in\R^n$ is the state vector.
The smooth property of $\bdf(\bdx,t)$ is essential to ensure the existence and uniqueness of the solution to \eqref{eq:sys} \cite[see, pp. 88-89]{Khalil:2002aa}.

\begin{figure}[!t]
    \centering
    \includegraphics[width=0.5\textwidth]{figs/lyaVSctrac.png}
    \caption{
        Difference between Lyapunov and contraction theory \cite[Fig. 1]{Tsukamoto:2021aa}.
        The Lyapunov theory investigates the convergence to a single point and the contraction theory does regarding a single trajectory.
    }
    \label{fig:lyaVSctrac}
\end{figure}

The biggest difference between the traditional Lyapunov theory and the contraction theory is that the contraction theory investigates the convergence of the state trajectory to a single trajectory (contraction behavior), while the Lyapunov theory focuses on the convergence of the state trajectory to a single point \ie see, Fig.~\ref{fig:lyaVSctrac}.
For this, motivated by the calculus of variations \cite[Chap. 4]{Kirk:2004aa}, \eqref{eq:sys} can be rewritten as differential dynamics using \textit{differential displacement} $\delta\bdx$ as follows:
\begin{equation}
    \ddtt\delta\bdx
    =
    \tpfpx(\bdx,t)
    \delta\bdx
    .
    \label{eq:diff_sys}
\end{equation}
For your information, $\delta\bdx$ is an infinitesimal displacement at \textit{fixed time}.

\subsubsection{Definitions}

Before we present the fundamental theorem of contraction theory, we introduce the following definitions.
One can re-visit this section while reading further.

\begin{definition}[see, Def. 2.2 \cite{Tsukamoto:2021aa}]  
    If any two trajectories $\mv\xi_1(t)$ and $\mv\xi_2(t)$ of \eqref{eq:sys} converge to a single trajectory, then the system \eqref{eq:sys} is said to be \textit{incrementally exponentially stable}, if $\exists C,\alpha>0$, subject to the following holds:
    \begin{equation}      
        \norm{\mv\xi_1(t)-\mv\xi_2(t)}
        \le 
        C
        \norm{\mv\xi_1(0)-\mv\xi_2(0)}
        \exp^{-\alpha t}
        ,\ \forall t\in\R_{\ge 0}
        .
    \end{equation}
    The result of Theorem \ref{thm:ctrac:main} equivalently implies the incremental exponential stability, since we have $\norm{\mv\xi_1(t)-\mv\xi_2(t)}= \norm{\int_{\mv\xi_1(t)}^{\mv\xi_2(t)}\delta\bdx(t)}$.
    \label{def:inc_exp_stable}
\end{definition}

\begin{definition}
    Let $\mm\Theta(\bdx,t)$ be a smooth coordinate transformation of $\delta\bdx$ to $\delta\bdz$, \ie $\delta\bdz = \Theta(\bdx,t)\delta\bdx$.
    Then, a symmetric continuously differentiable matrix $\mm M(\bdx,t):=\mm\Theta(\bdx,t)^\top\mm\Theta(\bdx,t)$ is said to be a \textit{metric} of the system \eqref{eq:sys}.
    \label{def:metric}
\end{definition}

\begin{definition}
    The covariant derivative of $\mv f(\bdx,t)$ in $\delta\bdx$ coordinate is represented as 
    \begin{equation}
        \mm F
        :=
        \left(
            \ddtt\mm\Theta
            +
            \mm\Theta\tpfpx
        \right)
        \mm\Theta^{-1}
        ,
    \end{equation}
    and is called the \textit{generalized Jacobian}.
    This can be easily derived by differentiating $\delta\bdz = \Theta(\bdx,t)\delta\bdx$ with respect to $t$, leading to $\ddtt\bdz=\mm F\bdz$.
    \label{def:gen_jac}
\end{definition}

\subsubsection{Fundamental Theorem of Contraction Theory}

The following theorem presents the fundamental theorem of contraction theory and corresponding necessary and sufficient condition for exponential convergence of the differential system \eqref{eq:diff_sys}.

\begin{theorem}[see, T. 2.1 \cite{Tsukamoto:2021aa}]
    If $
        \exists \mm{M}(\bdx,t)
        =
        \mm\Theta(\bdx,t)^\top
        \mm\Theta(\bdx,t)
        > 0, \forall \bdx,t
    $ where $\mm\Theta(\bdx,t)$ (see, Definition \ref{def:metric}) defines a smooth coordinate transformation of $\delta\bdx$ to $\delta\bdz$, \ie $\delta\bdz = \Theta(\bdx,t)\delta\bdx$, subject to the following equivalent conditions holds for $\exists\alpha\in\R_{>0},\ \forall \bdx,t$:
    \begin{subequations}
        \begin{align}
            \lambda_{\max} (
                \mm F(\bdx,t)
            )
            =
            \lambda_{\max} 
            \left(
                \left(    
                \ddtt \mm\Theta
                +
                \mm\Theta\tpfpx
                \right)
                \mm\Theta^{-1}
            \right)
            \le&
            -\alpha
            ,
        \label{eq:ctrac:metric:F}
            \\
            \ddtt \mm M
            +
            \mysym(\mm M\tpfpx)
            \le&
            -2\alpha\mm M
            ,
        \label{eq:ctrac:metric:M}
        \end{align}
        \label{eq:ctrac:metric}
    \end{subequations}
    where the arguments of $\mm M(\bdx,t)$ and $\mm F(\bdx,t)$ are omitted for simplicity, then, the system \eqref{eq:sys} is said to be contracting with an exponential rate $\alpha$, \ie all trajectories of \eqref{eq:sys} converge to a single trajectory.
    The converse is also true.
    \label{thm:ctrac:main}
\end{theorem}

\begin{proof}
    Taking time derivative of a differential Lyapunov function of $\delta\bdx$, $V=\delta\bdz^\top\bdz=\delta\bdx^\top\mm M\delta\bdx$, using the differential dynamics \eqref{eq:diff_sys}, we have
    \begin{equation}
        \begin{aligned}
            \ddtt V (\bdx, \delta\bdx, t)
            =&
            \delta\bdx^\top
            \left(
                \ddtt\mm M
                +
                \mysym(\mm M\tpfpx)
            \right)
            \delta\bdx
            =
            2
            \delta\bdz^\top
            \mm F
            \delta \bdz
            \\
            \le&
            -2\alpha
            \delta\bdx^\top
            \mm M
            \delta\bdx
            =
            -
            2\alpha
            \delta\bdz^\top
            \delta\bdz
            =
            -2\alpha V
            .
        \end{aligned}
    \end{equation}
    According to the comparison lemma (Lemma \ref{lem:comparison}), we have $V(t)\le V(0)e^{-2\alpha t}$, which then yields $\norm{\delta\bdz(t)}^2\le\norm{\delta\bdz(0)}^2e^{-2\alpha t}$ and $\norm{\delta\bdz(t)}\le\norm{\delta\bdz(0)}e^{-\alpha t}$.
    This implies that any infinitesimal displacement $\delta\bdz$ and $\delta\bdx$ converges to zero exponentially with rate $\alpha$.
    Note that the initial conditions are exponentially "forgotten" as time goes on.
    The proof of the converse can be found in \cite[Sec. 3.5]{LOHMILLER:1998aa}.
\end{proof}

It is notable that the unboundedness of the metric $\mm M(\bdx, t)$ does not create any problem in a technical sense.
This is because, the dynamics of $\mm M(\bdx, t)$ is linear with infinite escape time.
Therefore, it can be handled by renormalizing the metric $\mm M(\bdx, t)$.

\hfill

Theorem \ref{thm:ctrac:main} can also be proven by using the transformed squared length integrated over two arbitrary solutions of \eqref{eq:sys}.
The following theorem presents the alternative proof of Theorem \ref{thm:ctrac:main}.

\begin{theorem}[see, p. 688 \cite{LOHMILLER:1998aa}, T. 2.3 \cite{Tsukamoto:2021aa}]
    Let $\mv{\xi}_1(t)$ and $\mv{\xi}_2(t)$ be two solutions of \eqref{eq:sys}, and define the transformed squared length with $\mm M(\bdx,t)$ of Theorem \ref{thm:ctrac:main} as follows:
    \begin{equation}
        V_{sl}(\bdx,\delta\bdx,t)
        =
        \textstyle\int_{\mv{\xi}_1(t)}^{\mv{\xi}_2(t)}
        \norm{\delta\bdz}^2
        =
        \textstyle\int_0^1
        \pptfrac{\bdx}{\mu}^\top
        \mm M(\bdx,t)
        \pptfrac{\bdx}{\mu}
        \der\mu
        ,
        \label{eq:Vsl}
    \end{equation}
    where $\bdx$ is a smooth path parameterized as $\bdx(\mu=0,t)=\mv\xi_1(t)$ and $\bdx(\mu=1,t)=\mv\xi_2(t)$ by $\mu\in\{0,1\}$.
    Also, define the path integral with the transformation $\mm\Theta(\bdx,t)$ for $\mm M(\bdx,t)=\mm\Theta(\bdx,t)^\top\mm\Theta(\bdx,t)$ as follows:
    \begin{equation}
        V_{l}(\bdx,\delta\bdx,t)
        =
        \textstyle\int_{\mv{\xi}_1(t)}^{\mv{\xi}_2(t)}
        \norm{\delta\bdz}
        =
        \textstyle\int_{\mv{\xi}_1(t)}^{\mv{\xi}_2(t)}
        \norm{\mm\Theta(\bdx,t)\delta\bdx}
        .
        \label{eq:Vl}
    \end{equation}
    Then, \eqref{eq:Vsl} and \eqref{eq:Vl} satisfy the following inequality:
    \begin{equation}
        \norm{\mv\xi_1(t)-\mv\xi_2(t)}
        =
        \norm{
            \textstyle\int_{\mv{\xi}_1(t)}^{\mv{\xi}_2(t)}
            \delta\bdx
        }
        \le
        \tfrac{V_{l}}{\sqrt{\underbar{m}}}
        \le
        \sqrt{\tfrac{V_{sl}}{\underbar{m}}}
        ,
        \label{eq:xi_x_Vl_Vsl}
    \end{equation}
    where $\mm M(\bdx,t) \ge \underbar{m}\mm I_n,\ \forall \bdx, t$ for $\exists\underbar{m}\in\R_{>0}$ and Theorem \ref{thm:ctrac:main} can also be proven by using \eqref{eq:Vsl} and \eqref{eq:Vl} as a Lyapunov-like function, resulting in incremental exponential stability of the system \eqref{eq:sys} (see, Definition \ref{def:inc_exp_stable}).
    Note that the shortest path integral $V_l$ of \eqref{eq:Vl} with a parameterized state $\bdx$ (\ie $\inf{V_l}=\sqrt{\inf{V_{sl}}}$) defines the Riemannian distance and the path integral of a minimizing geodesic.
    \label{thm:ctrac:path_int}
\end{theorem}

\begin{proof}
    Note that $\mv{\xi}_i(t),\ \forall i\in\{1,2\}$ are parameterized $\bdx$.
    Therefore, left-side equailty of \eqref{eq:xi_x_Vl_Vsl} holds.
    The right-side inequality of \eqref{eq:xi_x_Vl_Vsl} can be proven by using the Cauchy-Schwarz inequality.

    On the other hands, by taking time derivative of $V_{sl}$ and $V_l$, we have $\ddtt V_{sl} \le -2\alpha V_{sl}$ and $\ddtt V_l \le -\alpha V_l$.
    As $\mm{M}(\bdx,t)$ is from Theorem \ref{thm:ctrac:main}, the incremental exponential stability of the system \eqref{eq:sys} is guaranteed using the comparison lemma of Lemma \ref{lem:comparison} as follows: 
    \begin{equation}
        \norm{\mv\xi_1(t)-\mv\xi_2(t)}
        \le
        \tfrac{V_{l}(0)}{\sqrt{\underbar{m}}} \exp(-\alpha t)
        .
    \end{equation}
\end{proof}

In conclusion, Theorem \ref{thm:ctrac:main} defines necessary and sufficient conditions of incremental exponential stability of the system \eqref{eq:sys}, and Theorem \ref{thm:ctrac:path_int} provides an alternative proof of Theorem \ref{thm:ctrac:main} using the path integral of a minimizing geodesic.

\hfill

We provide two simple examples of the contraction system.

\begin{example}[see, Ex. 2.2 \cite{Tsukamoto:2021aa}]
    Consider the following system:
    \begin{equation}
        \dot x = -x +\exp(t)
        .
    \end{equation}
    The differential dynamics of the system can be rewritten as \eqref{eq:diff_sys} with $\delta\bdx=\delta x$ as follows:
    \begin{equation}
        \ddtt\delta x
        =
        -\delta x
        .
    \end{equation}
    Using constraint \eqref{eq:ctrac:metric:M}, one can select the metric $\mm M=I$.
    Then, one can conclude that the system \eqref{eq:sys} is contracting with an exponential rate $\alpha=1$.
    However, the system \eqref{eq:sys} is not stable, since the solution of the system \eqref{eq:sys} is $x(t) = \tfrac{\exp(t)}{2}+(x(0)-\tfrac{1}{2})\exp(-t)$.
    \label{ex:div}
\end{example}

\begin{example}[see, Ex. 1 \cite{Jouffroy:2004aa}]
    Consider the following system:
    \begin{equation}
        \ddtt
        \begin{bmatrix}
            x_1\\
            x_2
        \end{bmatrix}
        =
        \begin{bmatrix}
            -1 & x_1\\
            -x_1 & -1
        \end{bmatrix}
        \begin{bmatrix}
            x_1\\
            x_2
        \end{bmatrix}
        .
    \end{equation}
    Using Lyapunov function $V=\tfrac{1}{2}(x_1^2+x_2^2)$, UGES (Uniformly Globally Exponentially Stable) can be easily proven.
    The differential dynamics of the system can be rewritten as \eqref{eq:diff_sys} with $\delta\bdx=\begin{bmatrix}\delta x_1 & \delta x_2\end{bmatrix}^\top$ as follows:
    \begin{equation}
        \ddtt
        \begin{bmatrix}
            \delta x_1\\
            \delta x_2
        \end{bmatrix}
        =
        \begin{bmatrix}
            -1+x_2 & x_1\\
            -2x_2 & -1
        \end{bmatrix}
        \begin{bmatrix}
            \delta x_1\\
            \delta x_2
        \end{bmatrix}
        .
    \end{equation}
    Even though the system is UGES, it is difficult to prove the contraction of the system, since the skew-symmetric structure of the system matrix is destroyed ub derivation process.

    Alternatively, one can use Lyapunov function $V=\tfrac{1}{2}((x_1-x_1')^2+(x_2-x_2')^2))$ where $x_1'$ and $x_2'$ denotes another solution.
    Hence, the contraction should be compared to an \textit{incremental} form of stability.
\end{example}

\hfill

However, as discussed in Example \ref{ex:div}, aforementioned theorems are limited to the convergence of the state trajectory to a single trajectory.
In the next section, we introduce the partial contraction theory to investigate the convergence of the state trajectory to a desired trajectory.

\subsubsection{Partial Contraction Theory}

The following theorem presents the partial contraction theory to ensure the convergence of the state trajectory to a desired trajectory.

\begin{theorem}[see, T. 1 \cite{Wang:2004aa}, T. 3 \cite{Jouffroy:2004aa}, T. 2.2 \cite{Tsukamoto:2021aa}]
    Consider a nonlinear system of the form 
    \begin{equation}
        \ddtt \bdx = \bdf(\bdx,\bdx,t)
        ,
    \end{equation}
    and assume that the virtual system
    \begin{equation}
        \ddtt\bdq = \bdf(\bdq,\bdx,t)
        ,
    \end{equation}
    is contracting with respect to $\bdq$.
    If a particular solution of the virtual $\bdq$-system verifies a smooth specific property, then all solutions of the original $\bdx$-system verifies this property exponentially.
    The original system is said to be partially contracting.
    \label{thm:ctrac:partial}
\end{theorem}

\begin{proof}
    The virtual, observer-like $\bdq$-system has two particular solutions, namely $\bdq(t) = \bdx(t),\ \forall t\in\R_{\ge0}$ and the solution with the specific property.
    Since all trajectories of the $\bdq$-system converge exponentially to a single trajectory, this implies that $\bdx(t)$ verifies the specific property exponentially.
\end{proof}

I am sure that you cannot easily undertand the word \textbf{specific property} in Theorem \ref{thm:ctrac:partial}.
I hope the following example helps you to understand the partial contraction theory.

\begin{example}[see, Ex. 4 \cite{Jouffroy:2004aa}]
    Consider the system 
    \begin{equation}
        \ddtt \bdx = -\mm D(\bdx) + \bdu,
    \end{equation}
    and reference system 
    \begin{equation}
        \ddtt \bdx_d = -\mm D(\bdx_d).
    \end{equation}
    Using control input $\bdu = -\mm K(\bdx)(\bdx-\bdx_d)+(\mm D(\bdx)-\mm D(\bdx_d))\bdx_d$, the virtual system can be rewritten as
    \begin{equation}
        \ddtt\bdq=-\mm D(\bdx)+\mm K(\bdx)(\bdq-\bdx_d)-\mm D(\bdx_d)\bdx_d
        .
    \end{equation}
    The particular solutions of the virtual system are $\bdq(t)=\bdx(t)$ and \textbf{specific property} $\bdq(t)=\bdx_d(t)$.
    Now, investigate the contraction of the virtual system with the differential dynamics of $\delta\bdq$ as follows: $\ddtt\delta\bdq=-(\mm D(\bdx)+\mm K(\bdx))\delta\bdq$.
    With contraction metric $\mm M=I$ (\ie $\mm D(\bdx)+\mm K(\bdx)>0$), one can conclude that the virtual system is contracting with an exponential rate $\alpha=1$.

    This implies that all solutions including two solutions (\ie $\bdq(t)=\bdx(t)$ and $\bdq(t)=\bdx_d(t)$), of the virtual system will converge to a single trajectory exponentially, leading to $\bdq(t)=\bdx(t)=\bdx_d(t)$.
    
    It is notable that the selection of the virtual system is crucial to simply obtain contraction metric $\mm M(\bdx,t)$.
    If we select the virtual system as $\ddtt \bdq = -(\mm D(\bdq)+\mm K(\bdq))(\bdq-\bdx_d)-\mm D(\bdx_d)\bdx_d$, then the contraction metric $\mm M$ is not easily obtained, since $\pptfrac{\mm K}{\bdq}$ and $\pptfrac{\mm D}{\bdq}$ are not zero matrices.
\end{example}
    
\subsubsection{Derterministic Perturbation}

Let us define deterministic prerturbated system by adding a bounded perturbation $\bdd(\bdx,t)$ to the system \eqref{eq:sys} as follows:
\begin{equation}
    \ddtt\bdx
    =
    \bdf(\bdx,t)
    +
    \bdd(\bdx,t)
    ,
    \label{eq:sys:pert}
\end{equation}
and two solutions of \eqref{eq:sys} $\mv\xi_1(t)$ and \eqref{eq:sys:pert} $\mv\xi_2(t)$.
By parameterzing virtual state $\bdq(\mu,t)$ with $\mu\in[0,1]$ such that $\bdq(\mu=0,t)=\xi_0$ and $\bdq(\mu=1,t)=\xi_1$, we have virtual system as follows:
\begin{equation}
    \ddtt\bdq = \bdf(\bdq(\mu,t),t)+d_\mu(\mu,\xi_1,t)
    ,
    \label{eq:sys:pert:virtual}
\end{equation}
where $d_\mu(\mu,\xi_1,t)=\mu d(\xi_1,t)$, whose particular solutions are $\bdq(\mu=0,t)=\xi_0$ and $\bdq(\mu=1,t)=\xi_1$.

The following theorem states the robustness of contracting system to the deterministic perturbation by extending Theorem \ref{thm:ctrac:path_int}.

\begin{theorem}[see, Eq. 15 \cite{LOHMILLER:1998aa}, Lemma 1 \cite{Dani:2015aa}, T. 2.4 \cite{Tsukamoto:2021aa}]
    If the system \eqref{eq:sys} satisfies \eqref{eq:ctrac:metric} of Theorem \ref{thm:ctrac:main}, then the path integral of the virtual system \eqref{eq:sys:pert:virtual} (\ie $V_l(\bdq,\delta\bdq,t)=\int_{\mv\xi_1}^{\mv\xi_2}\norm{\mm\Theta(\bdq,t)\delta\bdq}$), where $\mv{\xi}_1(t)$ and $\mv{\xi}_2(t)$ are solutions of \eqref{eq:sys} and \eqref{eq:sys:pert}, respectively, converges to a bounded error ball as long as $\mm\Theta\bdd\in \mathcal{L}_\infty$.
    Specifically, if $\exists \underbar m, \bar m\in\R_{>0}$ such that $\underbar m\mm I_n\le\mm M(\bdx,t)\le\bar m\mm I_n$, and $\bar d = \sup_{\bdx,t} \norm{\bdd(\bdx,t)}$, then the following inequality holds:
    \begin{equation}
        \norm{\mv\xi_1(t)-\mv\xi_2(t)}
        \le
        \tfrac{V_{l}(0)}{\sqrt{\underbar{m}}}
        \exp(-\alpha t)
        +
        \tfrac{\bar d}{\alpha}
        \sqrt{\tfrac{\bar m}{\underbar m}}
        (1-\exp(-\alpha t))
        ,
    \end{equation}
    where $V_l(t)=V_l(\bdq,\delta\bdq,t)$.
    \label{thm:ctrac:pert}
\end{theorem}

\begin{proof}
    Invoking $\ddtt(\pptfrac{\bdq}{\mu})=\pptfrac{\bdf}{\bdq}\pptfrac{\bdq}{\mu}+\bdd$ and $\mm M=\mm\Theta^\top\mm\Theta$, we have time derivative integrand of $V_l$ as follows:
    \begin{equation}
        \begin{aligned}
            \ddtt \norm{\mm\Theta\pptfrac{\bdq}{\mu}}
            =&
            \tfrac{1}{2}
            \left(
                \pptfrac{\bdq}{\mu}^\top
                \mm M
                \pptfrac{\bdq}{\mu}
            \right)^{-\tfrac{1}{2}}
            \left[
                \pptfrac{\bdq}{\mu}^\top
                \ddtt \mm M
                \pptfrac{\bdq}{\mu}
                +
                \pptfrac{\bdq}{\mu}^\top
                \mysym(\mm\Theta\tpfpx)
                \pptfrac{\bdq}{\mu}
                +
                \pptfrac{\bdq}{\mu}^\top
                2
                \mm M \bdd            
            \right]
            \\
            \le&
            \tfrac{1}{2}
            \left(
                \pptfrac{\bdq}{\mu}^\top
                \mm M
                \pptfrac{\bdq}{\mu}
            \right)^{-\tfrac{1}{2}}
            \left[
            -2\alpha 
            \pptfrac{\bdq}{\mu}^\top
            \mm M
            \pptfrac{\bdq}{\mu}
            +
            \pptfrac{\bdq}{\mu}^\top
            2
            \mm M \bdd            
            \right]
            \\
            \le &
            -\alpha
            \norm{\mm\Theta\pptfrac{\bdq}{\mu}}
            +
            \norm{\mm\Theta\pptfrac{\bdq}{\mu}}^{-1}
            \norm{\mm\Theta\pptfrac{\bdq}{\mu}}
            \norm{\mm\Theta \bdd}
            \\
            \le &
            -\alpha
            \norm{\mm\Theta\pptfrac{\bdq}{\mu}}
            +
            \norm{\mm\Theta \bdd}
            .
        \end{aligned}
    \end{equation}
    By integrating the above inequality with respect to $\mu$ from $0$ to $1$, we have $\ddtt V_l(t)=-\alpha V_l(t)+\sup_{\bdq,\mv\xi_1,t}\norm{\mm\Theta\bdd}$.
    Hence, by applying the comparison lemma (Lemma \ref{lem:comparison}), we have the following result as follows:
    \begin{equation}
        V_l(t)
        \le
        V_l(0)\exp(-\alpha t)
        +
        \tfrac{\sup_{\bdq,\mv\xi_1,t}\norm{\mm\Theta\bdd}}{\alpha}
        (1-\exp(-\alpha t))
        .
    \end{equation}
    By using the inequality \eqref{eq:xi_x_Vl_Vsl}, we have the desired result as follows:
    \begin{equation}
        \norm{\mv\xi_1(t)-\mv\xi_2(t)}
        \le
        \tfrac{V_{l}(0) \exp(-\alpha t)}{\sqrt{\underbar{m}}}
        +
        \tfrac{\bar d}{\alpha}
        \sqrt{\tfrac{\bar m}{\underbar m}}
        (1-\exp(-\alpha t))
        ,
    \end{equation}
    where $\underbar m=\inf_t\lambda_{\min}(\mm M)$ and $\bar m=\sup_t\lambda_{\max}(\mm M)$.
    It is notable that Theorem \ref{thm:ctrac:path_int} is a special case of Theorem \ref{thm:ctrac:pert} with $\bdd=\mv 0$.
\end{proof}

Since most of systems are subject to the perturbation, Theorem \ref{thm:ctrac:pert} will be utilized in the most of further cases.

\subsection{Basic Results of Contraction Theory for Stochastic Systems}

\color{red}
WILL BE UPDATED LATER.
\color{black}